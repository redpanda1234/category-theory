\documentclass[nocover]{pset}
\usepackage{tikz-cd}
\pagestyle{fancy}
\fancyhf{}
\lhead{Forest Kobayashi}
\chead{Algebraic Topology}
\rhead{Math 196 -- Fall, 2018}
\rfoot{\thepage\ of \pageref{LastPage}}
\setlength{\headheight}{15.2pt}
\setlength{\headsep}{10pt}
\lfoot{Friday, October 18th 2018}

\usepackage[normalem]{ulem} % [normalem] prevents the package from
                            % changing the default behavior of `\emph`
                            % to underline.

\titleformat{\section}
  {\LARGE \scshape}{\thesection.}{.5em}{\vspace{.5em}}

\titleformat{\subsection}
  {\Large \scshape}{\thesubsection.}{.5em}{}

\tikzstyle{titlerule}=[dash pattern=on \pgflinewidth off 2pt]
\usetikzlibrary{decorations.markings}

\usepackage{scalerel}
\DeclareMathOperator*{\csum}{\scalerel*{\#}{\sum}}

\newcommand{\wah}[1]{
  \tikz[decoration={markings, mark=between positions 0 and 1 step 3pt
    with { \draw [fill] (0,0) circle [radius=.5pt];}},
  baseline=(todotted.base)]{
  \node[inner sep=0pt,outer sep=0pt] (todotted) {#1};
  \path[postaction={decorate}] (todotted.south west) --
  (todotted.south east);}
}

\newcommand{\udot}[1]{%
    \tikz[baseline=(todotted.base)]{
        \node[inner sep=1pt,outer sep=0pt] (todotted) {#1};
        \draw[titlerule] (todotted.south west) -- (todotted.south east);
    }%
}%

\usepackage{upgreek}

\usetikzlibrary{
  knots,
  hobby,
  decorations.pathreplacing,
  shapes.geometric,
  calc
}

\tikzset{
  knot diagram/every strand/.append style={
    ultra thick,
    blue
  },
  show curve controls/.style={
    postaction=decorate,
    decoration={show path construction,
      curveto code={
        \draw [blue, dashed]
        (\tikzinputsegmentfirst) -- (\tikzinputsegmentsupporta)
        node [at end, draw, solid, blue, inner sep=2pt]{};
        \draw [blue, dashed]
        (\tikzinputsegmentsupportb) -- (\tikzinputsegmentlast)
        node [at start, draw, solid, blue, inner sep=2pt]{}
        node [at end, fill, blue, ellipse, inner sep=2pt]{}
        ;
      }
    }
  },
  show curve endpoints/.style={
    postaction=decorate,
    decoration={show path construction,
      curveto code={
        \node [fill, red, ellipse, inner sep=2pt] at (\tikzinputsegmentlast) {}
        ;
      }
    }
  }
}

\usepackage{caption}

\begin{document}

\begin{center}
  {\scshape \Large Basic Algebraic Topology}

  {\itshape Based on Kosniowski; Matveev}
\end{center}
\vspace{-.1cm}
\hrulefill

\section{Picking up where we left off}
\begin{adjustwidth}{1em}{1em}
  \subsection{Some Theorems}
  We'll breeze briskly through some more topology, then transition
  into knot theory topics.
  \begin{theorem}
    Let $Y$ be the quotient space of the topological space $X$
    determined by the surjective mapping $f : X \to Y$. If $X$ is
    compact Hausdorff and $f$ is closed then $Y$ is (compact)
    Hausdorff.
  \end{theorem}
  % \begin{proof}
  %   Let $y_1, y_2 \in Y$, with $y_1 \neq y_2$. Let $X_1 =
  %   f^{-1}(y_1),\ X_2 = f^{-1}(y_2)$. Since $f$ is closed, and every
  %   point of $X$ is closed, then for $X_1, X_2$, take any two
  %   representative elements $x_1$ and $x_2$, and use the fact that
  %   $f(\set{x_1}), f(\set{x_2})$ is the closed image of a close set,
  %   thus $\set{y_1}$ and $\set{y_2}$ are closed in $Y$. Then because
  %   $f$ is continuous, we get $X_1$ and $X_2$ are closed in $X$. Note
  %   too that they must be disjoint, because $y_1 \neq y_2$.
  % \end{proof}
  \begin{corollary}
    Let $X$ be a compact Hausdorff $G$-space with $G$ finite. Then
    $X/G$ is a compact Hausdorff space.
  \end{corollary}
  \begin{corollary}
    If $X$ is a compact Hausdorff space and $A$ is a closed subset of
    $X$ then $X/A$ is a compact Hausdorff space.
  \end{corollary}
  \end{adjustwidth}
  \section{All Together Now\ldots}
  \begin{adjustwidth}{1em}{1em}
  \subsection{Connectedness}~
  \begin{definition}
    Let $(X,\tau)$ a topological space. Then $X$ is said to be
    \emph{connected} iff the only clopen subsets are trivial. If $S
    \subseteq S$, then $S$ is said to be connected iff it is connected
    in the induced topology.

    Equivalently, $X$ is connected iff it cannot be expressed as the
    union of finitely many disjoint non-empty open subsets.
  \end{definition}
  \begin{theorem}
    Let $f : X \to Y$ be continuous, and suppose $X$ is connected.
    Then $Y$ is connected as well.
  \end{theorem}
  \begin{proof}
    Suppose, to obtain a contradiction, that $Y$ is disconnected. Then
    there exist nonempty open sets $U,V \subseteq Y$ with $U \cup V =
    Y$, and $U \cap V = \varnothing$. Since $U,V$ are open and $f$
    continuous, then $f^{-1}(U)$, $f^{-1}(V)$ are open in $X$.
    Furthermore, these are disjoint nonempty subsets of $X$ with
    $f^{-1}(U) \cup f^{-1}(V) = X$. Then $X$ is disconnected, a
    contradiction. Hence $Y$ is connected.
  \end{proof}
  \begin{theorem}
    Suppose that $\set{Y_j \MID j \in J}$ is a collection of connected
    subsets of a space $X$. If $\bigcap_{j \in J} Y_j \neq
    \varnothing$, then $Y = \bigcup_{j \in J} Y_j$ is connected.
  \end{theorem}
  \begin{proof}
    Suppose $U$ is a nonempty clopen subset of $Y$. Then $\exists j
    \in J \st U \cap Y_j \neq \varnothing$. Hence, let $J' = \set{j
      \in J \mid U \cap Y_j \neq \varnothing}$. Then $\forall j' \in
    J'$, we have $U \cap Y_{j'}$ is clopen in the induced topology.
    Since $Y_{j'}$ is connected, it follows that $U \cap Y_{j'} =
    Y_{j'}$. Hence $U = Y$, so $Y$ is connected.
  \end{proof}
  \begin{theorem}
    Let $X,Y$ be topological spaces. Then $X,Y$ are connected iff $X
    \times Y$ is connected.
  \end{theorem}
  \begin{proof}~
    \begin{iffproof}
      \item Suppose $X,Y$ are connected. $\forall x \in X,\ y \in Y$,
        $X \times \set{y} \cong X$ and $\set{x} \times Y \cong Y$ (and
        thus each is connected), and note $(X \times \set{y}) \cap
        (\set{x} \times Y) \neq \varnothing$, thus by theorem 1.3
        their union is connected. Now, let $y \in Y$ be fixed. Observe
        that
        \[
          X \times Y = \bigcup_{x \in X} (X \times \set{y}) \cap
          (\set{x} \times Y)
        \]
        Hence $X \times Y$ is connected.
      \item Suppose $X \times Y$ is connected. Then since the
        canonical projection maps are continuous, it follows that
        $X,Y$ are connected (continuous image of a connected set is
        connected).
    \end{iffproof}
  \end{proof}
  \subsection{Path Connectedness}

  \section{A Brief Discussion of Manifolds}

  \begin{definition}[Manifolds]
    Let $n \in \ZZ^{>0}$, and let $(M, \tau)$ be a topological space.
    Then $M$ is called a manifold iff $M$ is Hausdorff, and $\forall m
    \in M$, there exists a neighborhood $N$ of $m$ such that $N \cong
    \mathring{D}^n = \set{x \in \RR^n \MID \norm{x} < 1}$.
  \end{definition}
  \begin{definition}[Connected Sum]
    Let $S_1$, $S_2$ be compact connected 2-manifolds (surfaces), and
    let $D_1 \subseteq S_1$, $D_2 \subseteq S_2$ with $D_1, D_2 \cong
    D^2$. Let $h_1 : D_1 \to D^2$, and $h_2 : D_2 \to D^2$ be
    homeomorphisms. Then define $\sim$ to be an equivalence relation
    such that $x \sim h^{-1}_2 h_1(x)$ iff $x \in \partial D_1$, and
    $x \sim x$ otherwise. Then $S_1 \# S_2$ is given by
    \[
      \frac{(S_1 - \mathring{D_1}) \cup (S_2 - \mathring{D_2})}{\sim}
    \]
  \end{definition}
  \begin{definition}
    Call a surface $S^2$ \emph{orientable} if it contains no
    M\"{o}bius strip, and \emph{non-orientable} otherwise. Then for $m
    \geq 0, n \geq 1$, we call
    \[
      S^2 \# \pn{\csum_{i=1}^m T} = S \# mT
    \]
    the \emph{standard orientable surface of genus $m$}, and
    \[
      S^2 \# \pn{\csum_{i=1}^n \RR P^2} = S \# n\RR P^2
    \]
    the \emph{standard non-orientable surface of genus $n$}.
  \end{definition}

\end{adjustwidth}

\end{document}
