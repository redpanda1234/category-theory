\documentclass[nocover]{pset}
\usepackage{tikz-cd}
\pagestyle{fancy}
\fancyhf{}
\lhead{Forest Kobayashi}
\chead{Algebraic Topology}
\rhead{Math 196 -- Fall, 2018}
\rfoot{\thepage\ of \pageref{LastPage}}
\setlength{\headheight}{15.2pt}
\setlength{\headsep}{10pt}
\lfoot{Friday, October 18th 2018}

\usepackage[normalem]{ulem} % [normalem] prevents the package from
                            % changing the default behavior of `\emph`
                            % to underline.

\titleformat{\section}
  {\LARGE \scshape}{\thesection.}{.5em}{\vspace{.5em}}

\titleformat{\subsection}
  {\Large \scshape}{\thesubsection.}{.5em}{}

\tikzstyle{titlerule}=[dash pattern=on \pgflinewidth off 2pt]
\usetikzlibrary{decorations.markings}

\newcommand{\wah}[1]{
  \tikz[decoration={markings, mark=between positions 0 and 1 step 3pt
    with { \draw [fill] (0,0) circle [radius=.5pt];}},
  baseline=(todotted.base)]{
  \node[inner sep=0pt,outer sep=0pt] (todotted) {#1};
  \path[postaction={decorate}] (todotted.south west) --
  (todotted.south east);}
}

\newcommand{\udot}[1]{%
    \tikz[baseline=(todotted.base)]{
        \node[inner sep=1pt,outer sep=0pt] (todotted) {#1};
        \draw[titlerule] (todotted.south west) -- (todotted.south east);
    }%
}%

\usepackage{upgreek}

\usetikzlibrary{
  knots,
  hobby,
  decorations.pathreplacing,
  shapes.geometric,
  calc
}

\tikzset{
  knot diagram/every strand/.append style={
    ultra thick,
    blue
  },
  show curve controls/.style={
    postaction=decorate,
    decoration={show path construction,
      curveto code={
        \draw [blue, dashed]
        (\tikzinputsegmentfirst) -- (\tikzinputsegmentsupporta)
        node [at end, draw, solid, blue, inner sep=2pt]{};
        \draw [blue, dashed]
        (\tikzinputsegmentsupportb) -- (\tikzinputsegmentlast)
        node [at start, draw, solid, blue, inner sep=2pt]{}
        node [at end, fill, blue, ellipse, inner sep=2pt]{}
        ;
      }
    }
  },
  show curve endpoints/.style={
    postaction=decorate,
    decoration={show path construction,
      curveto code={
        \node [fill, red, ellipse, inner sep=2pt] at (\tikzinputsegmentlast) {}
        ;
      }
    }
  }
}

\usepackage{caption}

\begin{document}

\begin{center}
  {\scshape \Large Basic Algebraic Topology}

  {\itshape Based on Kosniowski; Matveev}
\end{center}
\vspace{-.1cm}
\hrulefill

\section{Picking up where we left off}
\begin{adjustwidth}{1em}{1em}
  \subsection{Compactness}
  Last time, we finished by giving some basic definitions of
  comapctness, and whatnot. We'll begin with a small exercise to shake
  some of the cobwebs loose.
  \begin{enumerate}
    \item Suppose that $X$ has the finite complement topology. Show
      that $X$ is compact. Show that each subset of $X$ is compact.
    \item Prove that a topological space is compact if and only if
      whenever $\set{C_j \MID j \in J}$ is a collection of closed sets
      with $\bigcap_{j\in J} C_j = \varnothing$ then there is a finite
      subcollection $\set{C_k \MID k \in K}$ such that $\bigcap_{k \in
        K} C_k = \varnothing$.
    \item Let $\mc F$ be the topology on $\RR$ defined by $U \in \mc
      F$ iff $\forall s \in U, \exists t > s \st [s,t) \subseteq U$.
      Prove that the subset $[0,1]$ of $\mc F$ is not compact.
  \end{enumerate}
  Now
  \begin{enumerate}
    \item Let $U = \set{U_i \mid i \in I}$ be an open cover of $X$.
      Let $U_i \in U$. Then $X - U_i$ is finite. For every $x_j \in X
      - U_i$, $\exists U_j \in U \st x \in U_j$ (because $U$ is a
      cover). Then the set consisting of $U_i$ and the $U_j$ is a
      finite subcover, thus $X$ is compact. Let $Y \subseteq X$. Then
      let $V = \set{V_k \mid k \in K}$ be an open cover of $Y$.
      Proceed an analogous argument to the above to obtain $Y$
      compact.
    \item Let $X$ be a topological space, and suppose $X$ is compact.
      Let $C = \set{C_j \MID j \in J, C_j = \ol{C_j}}$ (i.e., the
      $C_j$ are closed), and suppose
      \[
        \bigcap_{j \in J} C_j = \varnothing.
      \]
      By De Morgan's laws,
      \[
        \bigcup_{j \in J} X - C_j = X
      \]
      since $C_j$ are all closed, then $X - C_j$ are open, hence this
      is an open cover of $X$, and there exists a finite subcover.
      Apply De Morgan's laws again to yield the desired result.
    \item Let $\varepsilon > 0$ be given. Let $U$ be given by the open
      cover
      \[
        U = % \underbrace{
          \set{ \left[ 1 - \frac{1}{2^i}, 1 - \frac{1}{2^{i +
                1}}\right) \MID i = 0, 1, \ldots }% }_{U_1}
      \cup % \underbrace{
        \set{\left[1, 1 + \varepsilon \right)}% }_{U_2},
      \]
      and note that all the sets in $U$ are disjoint, and that they
      cover $[0,1]$. From disjointness, it follows there is no finite
      subcover.
  \end{enumerate}
  \subsection{A brief review of projections}
  On our first pass through, we didn't treat projection maps in a lot
  of depth, so we'll very briefly revisit them here. \clearpage
  \begin{definition}[Projection Maps]
    Let $X, Y$ be topological spaces. Then define $\pi_X : X \times Y
    \to X$, $\pi_Y : X \times Y \to Y$ by
    \[
      \pi_X(x,y) = x \qquad \qquad \pi_Y(x,y) = y
    \]
    $\pi_X$ and $\pi_Y$ are referred to as the \emph{product
      projections}. Note that both are continuous.
  \end{definition}
  \section{Compactness, Continued}
  \begin{theorem}
    Let $(X, \tau)$ be a topological space, and let $S \subseteq X$.
    Then $S$ is compact in $(X, \tau)$ iff $S$ is compact under the
    induced topology.
  \end{theorem}
  \begin{proof}
    Forwards direction is trivial. For the backwards direction,
    suppose $S$ is compact in the induced topology. Let $U = \set{U_i
      \MID i \in I}$ be an open cover of $S$ in $(X, \tau)$. Then $V =
    \set{V_i = U_i \cap S \MID i \in I}$ is an open cover of $S$ in
    the induced topology, and hence by compactness there exists a
    finite subcover $V' = \set{V_{i_k} \MID i_k \in I, k=1,\ldots,n}$.
    Now, take $U' = \set{U_{i_k} \MID i_k \in I, k =1,\ldots, n}$.
    Then $U'$ is a finite subcover of $U$. Since $U$ was taken to be
    arbitrary, this implies $S$ is comapct.
  \end{proof}
  In the metrizable topologies we encountered in Real Analysis, we
  proved that continuous functions preserve compactness. However, we
  will now show that the same result holds in a general topological
  space.
  \begin{theorem}[Continuity and Compactness]
    Let $f : (X,\tau) \to (Y,\upsilon)$ be a continuous map. Let $S
    \subseteq X$ be a compact subspace. Then $f(S)$ is compact in $Y$.
  \end{theorem}
  \begin{proof}
    Let $V = \set{V_i \mid i \in I} \subseteq \upsilon$ be an open
    cover of $f(S)$. Because $f$ is continuous, $U = \set{f^{-1}(V_i)
      \MID i \in I}$ is a collection of open sets covering
    $f^{-1}(f(S)) \supseteq S$. Since $S$ is compact, there exists a
    finite subcover $U' = \set{f^{-1}(V_{i_k}) \MID i_k \in I, k =
      1,\ldots,n}$ covering $f^{-1}(f(S))$. Then $V' = \set{V_{i_k}
      \mid i_k \in I, k = 1,\ldots,n}$ is a finite subcover of $V$.
    Thus $f(S)$ is compact in $Y$.
  \end{proof}
  By virtue of the properties of continuous functions that we proved
  last time, some nice results follow immediately:
  \begin{corollary}
    ~
    \begin{enumerate}
      \item Any closed interval in $\RR$ is compact.
      \item If $X$ and $Y$ are homeomorphic, then $X$ is compact iff
        $Y$ is.
      \item If $X$ is compact, and $Y$ is any set, then $Y$ with the
        quotient topology induced by $f : X \to Y$ is compact.
    \end{enumerate}
  \end{corollary}
  For completeness, we list some closure properties of compact
  subspaces:
  \begin{theorem}
    Let $(X, \tau)$ be a topological space. Let $S = \set{S_i \MID i
      \in I} \subseteq$ be the collection of compact subspaces of $X$.
    Then
    \begin{enumerate}
      \item If $S_1, S_2 \in S$, then $S_1 \cup S_2 \in S$ (union of
        two compact subspaces is compact). It follows by induction
        that any finite union of compact subspaces is compact.
      \item It is not the case that in an arbitrary topological space,
        an arbitrary intersection of compact spaces is compact (we
        need Hausdorffness). But for finite intersections, things work
        out.
    \end{enumerate}
  \end{theorem}
  % \begin{proof}
  %   \begin{enumerate}
  %     \item Trivial
  %     \item Let
  %       \[
  %         Y = \bigcap_{k \in K} S_k,
  %       \]
  %       where $S_k \in S$, and $K$ is finite. Then let $U = \set{U_i
  %         \mid i \in I}$ be an open cover of $Y$.
  %   \end{enumerate}
  % \end{proof}
  \begin{theorem}
    Let $(X, \tau)$ be a compact topological space, and let $S
    \subseteq X$ be closed. Then $S$ is compact.
  \end{theorem}
  \begin{proof}
    Let $U = \set{U_i \mid i \in I}$ be an open cover of $S$. Let $U_0
    = X - S$. Then $U_0$ is open, and $U \cup \set{U_0}$ covers $X$.
    Then since $X$ is compact, there exists a finite subcover $U' =
    \set{U_i \mid i \in I \cup \set{0}}$. Take $U'' = U' - \set{U_0}$
    to obtain a finite subcover of $U$.
  \end{proof}
  I'm proud to have written this proof without looking at the one in
  the book at all, only to find later that they're basically
  identical.
  \begin{theorem}
    Let $X, Y$ be topological spaces. Then $X, Y$ are compact iff $X
    \times Y$ is compact.
  \end{theorem}
  \begin{proof}~
    \begin{enumerate}
      \item[($\Rightarrow$):] Suppose $X,Y$ are compact. WTS $X \times
        Y$ is compact as well. Let $W = \set{W_i \mid i \in I}$ be an
        open cover of $X \times Y$. Note that $\forall y \in Y$, $X
        \times \set{y}$ is homeomorphic to $X$.
      \item[($\Leftarrow$):] Suppose $X \times Y$ is compact. Let $U =
        \set{U_i \MID i \in I}$ be an open cover of $X$, and $V =
        \set{V_j \MID j \in J}$ be an open cover of $Y$. Then $W$
        given by
        \[
          W = \set{\bigcup_{k \in K} W_k \MID W_k \in U \times V}
        \]
        is an open cover of $X \times Y$, and thus admits a finite
        subcover:
        \[
          W' = \set{W_\ell \mid \ell \in L;\ \abs{L} < \infty}
          \subseteq W
        \]
        Apply a similar trick something something boom
    \end{enumerate}
  \end{proof}
  \section{Hausdorff Spaces}
  Hausdorffness is an important property in Topology that essentially
  allows us to separate things from each other (our space is not
  ``infinitely bunched-up'' somewhere).
  \begin{definition}
    Let $(X, \tau)$ be a topological space. Then call $X$
    \emph{Hausdorff} iff for all $x, y \in X$ such that $x \neq y$,
    there exist open sets $U_x, U_y$ with $x \in U_x$, $y \in U_y$,
    and $U_x \cap U_y = \varnothing$.
  \end{definition}
  Note that by a simple $\varepsilon/2$ argument, it follows that all
  metrizable spaces are Hausdorff.
  \begin{definition}[$T_k$ spaces]
    For $k = 0, 1, 2, 3, 4$, call $X$ a $T_k$ space if it satisfies
    the $k$-th condition below (indexing starts at 0):
    \begin{enumerate}[label=$T_\arabic*$:]
      \item[$T_0$:] For all $x,y \in X$ ($x \neq y$), there is an open
        set $U$ containing one but not the other (i.e., $x \in U$ and
        $y \not \in U$, or $y \in U$ and $x \not \in U$).
      \item For all $x,y \in X$ ($x \neq y$), there are open sets
        $U,V$ such that $x \in U$, $y \in V$, and $x \not \in V$, $y
        \not\in U$.
      \item For all $x,y$ in $X$ ($x \neq y$), there are open sets $U,
        V$ such that $x \in U$, $y \in V$, and $U \cap V =
        \varnothing$ (there are disjoint neighborhoods about $x$ and
        $y$).
      \item $X$ is $T_1$, and for all closed subsets $F$ and points $x
        \not \in F$, there exist open sets $U,V$ such that $F
        \subseteq U$, $x \in V$, and $U \cap V = \varnothing$.
      \item $X$ is $T_1$, and for all pairs of disjoint closed
        subsetes $F_1$, $F_2$, there exist open sets $U$, $V$ such
        that $F_1 \subseteq U$, $F_2 \subseteq V$
    \end{enumerate}
  \end{definition}
  Naturally, if $X$ and $Y$ are homeomorphic topological spaces, and
  $X$ is $T_k$, then $Y$ is $T_k$ as well. As an exercise, we
  construct spaces that are $T_j$ (for $j = 0, \ldots, 4$) that are
  not $T_{i > j}$.
  \begin{enumerate}[label=($X_\arabic*)$:]
    \item[($X_0$):] Let $X_0 = (\RR^{\geq 0}, \tau)$, where $\tau =
      \set{\bkpn{0,t} \MID t \in \RR^{\geq 0}}$. Note that $\tau$ is
      indeed a topology on $\RR^{\geq 0}$. Note $X_0$ is not $T_1$.
    \item Let $X_1 = (X, \tau)$ where
  \end{enumerate}
  \begin{theorem}
    A space $X$ is $T_1$ iff every point of $X$ is closed.
  \end{theorem}
  \begin{proof}
    Suppose $(X, \tau)$ is $T_1$. Let $x \in X$ be arbitrary, and let
    $y \in X- \set{x}$. Then $\exists U_y \in \tau$ with $y \in U_y$,
    but $x \not \in U_y$. Hence
    \[
      \bigcup_{y \in X - \set{x}} U_y = X - \set{x}
    \]
    is open, and so $\set{x}$ is closed.

    Now suppose $\set{x}$ is closed. Then $T_1$ follows immediately.
  \end{proof}
  An important theorem:
  \begin{theorem}
    Let $A$ be a compact subset of a Hausdorff space $X$. Then $A$ is
    closed.
  \end{theorem}
  \begin{proof}
    We define the following (bizarre) open cover. For all $a \in A$,
    for all $x \in X - A$, there exist disjoint open sets $U_a$, $V_a$
    such that $a \in U_a$, and $x \in V_a$. Then $U = \set{U_a}$ is an
    open cover of $A$, and thus contains a finite subcover $U' =
    \set{U_{a'} \MID a' \in \mc{A} \subseteq A, \abs{\mc{A}} <
      \infty}$, and corresponding $V'$. Then note that
    \[
      V_x = \bigcap_{a' \in A} V_{a'}
    \]
    is an open set (closure under finite intersections) such that $V_x
    \cap A = \varnothing$. Hence
    \[
      X - A = \bigcup_{x \in X} V_x
    \]
    and so $X - A$ is open. Then $A$ is closed.
  \end{proof}
  Now, it is time for another Very Important Theorem\texttrademark.
  \begin{theorem}
    Let $(X, \tau)$, $(Y, \upsilon)$ be topological spaces, with $X$
    compact and $Y$ Hausdorff. Let $f : X \to Y$ be continuous. Then
    $f$ is a homeomorphism iff $f$ is a bijection.
  \end{theorem}
  \begin{proof}~
    \begin{iffproof}
      \item Suppose $f$ is a bijection. Then $f^{-1}$ exists, and
        $ff^{-1} = \id{Y}$, $f^{-1}f = \id{X}$. WTS $f^{-1}$ is
        continuous. Let $S \subseteq X$ be closed. Then $S$ is
        compact, hence $\pn{f^{-1}}^{-1}(S) = f(S)$ is compact in $Y$.
        Thus $f(S)$ is closed as well, hence $f^{-1}$ is continuous.
        Thus $f$ is a homeomorphism.
      \item Suppose $f$ is a homeomorphism. Then $f$ is a bijection.
    \end{iffproof}
  \end{proof}
  Thus, we see that
\end{adjustwidth}

\end{document}
