\documentclass[nocover]{pset}
\usepackage{tikz-cd}
\pagestyle{fancy}
\fancyhf{}
\lhead{Forest Kobayashi}
\chead{Basic Category Theory}
\rhead{Math 196 -- Fall, 2018}
\rfoot{\thepage\ of \pageref{LastPage}}
\setlength{\headheight}{15.2pt}
\setlength{\headsep}{10pt}
\lfoot{Thursday, October 4th 2018}

\usepackage[normalem]{ulem} % [normalem] prevents the package from
                            % changing the default behavior of `\emph`
                            % to underline.

\titleformat{\section}
  {\LARGE \scshape}{\thesection.}{.5em}{\vspace{.5em}}

\titleformat{\subsection}
  {\Large \scshape}{\thesubsection.}{.5em}{}

\tikzstyle{titlerule}=[dash pattern=on \pgflinewidth off 2pt]
\usetikzlibrary{decorations.markings}

\newcommand{\wah}[1]{
  \tikz[decoration={markings, mark=between positions 0 and 1 step 3pt
    with { \draw [fill] (0,0) circle [radius=.5pt];}},
  baseline=(todotted.base)]{
  \node[inner sep=0pt,outer sep=0pt] (todotted) {#1};
  \path[postaction={decorate}] (todotted.south west) --
  (todotted.south east);}
}

\newcommand{\udot}[1]{%
    \tikz[baseline=(todotted.base)]{
        \node[inner sep=1pt,outer sep=0pt] (todotted) {#1};
        \draw[titlerule] (todotted.south west) -- (todotted.south east);
    }%
}%

\usepackage{caption}

\begin{document}

\begin{center}
  {\scshape \Large Basic Algebraic Topology}

  {\itshape Based on Notes by Sergey Matveev}
\end{center}
\vspace{-.1cm}
\hrulefill
\section{Introduction}
\begin{adjustwidth}{1em}{1em}
  First, a motivating quote.
  \begin{quote}
    ``Point set topology is a disease from which later generations
    will regard themselves as having recovered'' -Henri Poincar\'{e}
  \end{quote}
  As it turns out, lots of topics in topology can be simplified by
  attaching algebraic constructs to different topological spaces, and
  proving that certain properties of our group (or what have you)
  correspond naturally to properties of our topological space. The
  vehicle by which we navigate between the two is, as one might
  expect, Category Theory. First, we give a brief summary of
  basic concepts in point-set topology, before moving into the
  Homology Theory presentation given in Matveev.

  \subsection{Basic Point-Set Topology}
  As in most branches of mathematics, our object of study here will be
  some collection of sets, together with some \emph{structure} we can
  associate with them. In Elementary Algebra, this takes the form of
  \emph{group} and \emph{ring} operations, and later the respective
  homomorphisms preserving them. In Elementary Analysis, it (loosely
  speaking) took the form of a \emph{distance metric}, and the
  properties it bestowed on sets. Analogous to our study of
  homomorphisms in Algebra, we often studied \emph{continuous
    functions} in Analysis, and the properties of sets that they
  preserved. Note the resemblance between the two expressions:
  \begin{minipage}[H]{.49\linewidth}
    \begin{figure}[H]
      \centering
      \[
        \varphi(g_1 \oplus g_2) = \varphi(g_1) \otimes \varphi(g_2)
      \]
      \caption*{A homomorphism $\varphi : G \to H$}
    \end{figure}
  \end{minipage}
  \begin{minipage}[H]{.49\linewidth}
    \begin{figure}[H]
      \centering
      \[
        d(x,y) < \delta \implies d'(f(x), f(y)) < \varepsilon
      \]
      \caption*{A continuous function $f : (E,d) \to (E', d')$}
    \end{figure}
  \end{minipage}

  while the analogy doesn't hold exactly, in both cases, we have some
  particular class of functions such that structure in one space is
  preserved in the image. In the case of homomorphisms, the group
  operation in the first group is ``respected'' by the homomorphism
  upon mapping into the second. In the case of continuous functions,
  our equation is essentially stating that we don't ``tear'' our
  starting space at all. This is best visualized by thinking about our
  continuous functions not as their \emph{graphs} (as we are often
  used to), but rather as \emph{maps} that deform the input domain in
  various manners to yield the image. As an example, one might think
  of the function $f(x) = x^2$ as the action of folding $\RR$ on
  itself, and stretching the edges out towards infinity (this is often
  a strategy employed in visualizing complex-valued functions).

  One might wonder what sorts of interesting discoveries we could make
  by generalizing our starting premises on the right-hand-side, so
  that we could make our questions more similar to those on the left.
  That is, similarly to how we defined distance metrics so as to
  generalize the \emph{key} properties of Euclidean distance, so too
  will we generalize the idea of \emph{continuity of a function}. This
  is the central idea of basic topology. Now, all we need is a good
  place to start. Recall the following theorem of Analysis:
  \begin{definition}
    Let $(E,d)$ and $(E', d')$ be metric spaces. Then a function $f :
    E \to E'$ is said to be \emph{continuous} iff for all open sets $U
    \subseteq M_2$, we have $f^{-1}(U) \text{ is open in } M_1$
  \end{definition}
  Of course, this makes no guarantees about the image of an open set
  being open in the codomain. Really, we can make our image as
  ``jagged'' as we want (within reason), provided we fold and deform
  our domain in a smooth way. But it does indicate to us open sets
  appear to be intimately tied to the idea of ``smooth'' deformations
  --- and that it might be fruitful to pursue an understanding of our
  space that does not depend on the details of a particular metric,
  but rather just on relationships between open sets. Hence, we define
  a topology as follows:
  \begin{definition}
    Let $X$ be a set, and let $\mc U$ be a collection of subsets of
    $X$ satisfying the following:
    \begin{enumerate}[label=(\roman*)]
      \item $\varnothing \in \mc U$, $X \in \mc U$.
      \item For all $U_1, U_2 \in U$, $U_1 \cap U_2 \in \mc U$
        (closure under finite intersections).
      \item For any subset $\set{U_i \mid i \in I} \subseteq \mc U$,
        we have
        \[
          \bigcup_{i \in I} U_i = \bm{U} \in \mc U \qquad
          \text{(closure under arbitrary unions)}
        \]
    \end{enumerate}
    then $\mc U$ is called a \emph{topology} for $X$, and $(X, \mc U)$
    is called a \emph{topological space}. We call the elements of $\mc
    U$ the \emph{open sets of} $(X, \mc U)$.
  \end{definition}
  note that a topology is thus a particular kind of \emph{algebra of
    sets} under the binary operations $\cup, \cap$, with identity
  $\varnothing$ for $\cup$, and $X$ for $\cap$. Note that $(\cup,
  \varnothing)$, $(\cap, X)$ are duals of each other, in the sense
  that for any sentence $S$ built out of atomic propositions of our
  set algebra, if $S$ is true, then the statement we obtain by
  \begin{enumerate}[label=\arabic*.]
    \item Replacing each $\cup$ with $\cap$ and each $\cap$ with
      $\cup$,
    \item Interchanging each $\varnothing$ and $X$, and
    \item Reversing inclusions
  \end{enumerate}
  must also be true. Note that if we replace ``arbitrary unions'' with
  ``countable unions'', and further require closure under
  complementation, then we obtain a $\sigma$-algebra.

  As it turns out, this definition of a topology is more general than
  that given by distance metrics. Whereas every distance metric gives
  rise to a topology, there are topologies that are not
  \emph{metrizable}, meaning they do not arise from any metric on a
  set. We list a few common topologies. Let $(X, \mc U)$ be a
  topological space. Then
  \begin{enumerate}[label=\arabic*.]
    \item If $\mc U = \set{\varnothing, X}$, we call $\mc U$ the
      \emph{concrete} or \emph{indiscrete topology}.
    \item If $\mc U = \mc P(X)$ (i.e., every subset of $X$ is open),
      then we call $\mc U$ the \emph{discrete topology}. Note the
      analog to the discrete metric.
    \item Suppose $\mc U = \set{\varnothing, X} \cup \set{U \subseteq
        X \MID \abs{\ol{U}} < \infty}$. That is, $X, \varnothing$, and
      all subsets of $X$ with finite compliment. Then call $\mc U$ the
      \emph{finite complement topology}.
    \item
  \end{enumerate}
  our definition of continuity follows identically to in a metric
  space:
  \begin{definition}
    A function $f : X \to Y$ between two topological spaces is said to
    be \emph{continuous} if for every oepn set $U \subset Y$, the
    inverse image $f^{-1}(U)$ is open in $X$. The same holds for
    closed subsets. Continuous functions are composable to yield
    another continuous function. If $f$ is bijective with a continuous
    inverse, then call $f$ a \emph{homeomorphism}.
  \end{definition}

\end{adjustwidth}

\end{document}
